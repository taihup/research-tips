\documentclass[a4paper,	11pt]{article}

% add latex preamble
\input{./common/latex_preamble}

% add math preamble
\input{./common/math_preamble}

% changes margins
\usepackage[a4paper, total={7.5in, 10.6in}]{geometry}

% bibliography
\addbibresource{./common/bibliography.bib}

\begin{document}

\title{Guía de Escritura Científica}
\author{Taihú Pire}


\maketitle

\tableofcontents

\section{Introducción}

Este documento es una guía para hacer propuestas, informes, tesinas, tesis, publicaciones y presentaciones.

\section{Comentarios generales}
%
\begin{itemize}
    \item El documento en {\LaTeX} debe compilar sin errores ni warnings.

    \item Ser detallado y claro en el texto, repetir palabras hace que sea fácil de leer.

    \item Redactar de manera formal: no utilizar ``etc'' por ejemplo, en vez utilizar ``entre otros''.

    \item Mantener una terminología. Ejemplo: usar V-SLAM o SLAM visual, no ambas.

    \item Cada capítulo debe tener un texto introductorio que cuente de qué va el capítulo, qué se presenta o describe.
    
    \item Todas las ecuaciones (incluso las que están centradas) deben terminar con punto (.) o con coma (,). En el caso de coma y punto y seguido no debe haber sangría en la oración que sigue a la ecuación.

    \item Para un término en inglés Ej: \emph{keyframe} utilizar itálica únicamente en la primera ocurrencia de la palabra, luego no se usa más itálica para ese término.

    \item Para las unidades o números con unidades utilizar el paquete \lstinline{siunitx} de {\LaTeX}. Ej: \SI{5}{\meter} o \si{\meter} para unidades solas.

    \item Remover citas que no se referencien en el texto, esto hace que el .bib quede más limpio.
\end{itemize}

\section{Tesinas}
\subsection{Links de interés}
\begin{itemize}
    \item Sitio Comunidades de la Tesina: \url{https://comunidades.campusvirtualunr.edu.ar/course/view.php?id=6742}
    \item Página oficial de la Tesina de LCC: \url{https://dcc.fceia.unr.edu.ar/es/lcc/tesinas-grado}
\end{itemize}


\subsection{Ejemplos de source de Tesinas}
\begin{itemize}
    \item Tesina de Ismael Ait realizada en LyX: \url{https://github.com/iait/tesina}
    \item Tesina de Bruno Zanotti realizada en {\LaTeX}: \url{https://repositorio.cifasis-conicet.gov.ar/bzanotti/tesina}
\end{itemize}

\subsection{Estructura General de una Tesina}
Portada\\
Abstract\\
1. Introducción\\
1.1 Motivación\\
1.2 Objetivos\\
1.3 Trabajos publicados durante la tesina (si los hubiere)\\
1.4 Estructura de la tesina\\
2. Trabajo Relacionado\\
3. Conceptos Previos\\
4. Método\\
5. Experimentos $\leftarrow$ \textbf{Este capítulo es el primero que debe escribirse}\\
6. Conclusiones\\
6.1 Trabajo Futuro\\
Apéndices (si los hubiere)\\
Referencias

\section{Estructura general de un Abstract}
Cada oración del abstract tiene que decir algo puntual:\\
La 1ra oración puede ser la que pone contexto.\\
La 2da oración habla sobre el problema que se intenta abordar.\\
La 3ra oración describe el método.\\
La 4ta oración de cómo el método es validado.\\
(El abstract de la tesina puede ser un poco más extenso)


\section{Comentarios sobre Trabajo Relacionado}
\begin{itemize}
    \item Describir los trabajos relacionados que se referencian: características principales, ventajas, desventajas, etc.
    \item Ver el estado del arte de los papers e inspirarse en ellos. Ver qué trabajos citan y cómo los describen.
\end{itemize}

\subsection{Búsqueda y Análisis de Papers}
    \begin{itemize}
        \item Búsqueda Avanzada de Google Scholar \url{https://scholar.google.com/}
        \item Mendeley: manejador de papers \url{https://www.mendeley.com/}
        \item \url{https://www.connectedpapers.com/}: grafo para búsqueda del estado del arte
        \item NotebookLM: Ayuda en la comprensión \url{https://notebooklm.google.com/}
        \item Ranking de Google de revistas y conferencias de Robótica: \url{https://scholar.google.com/citations?view_op=top_venues&hl=es&vq=eng_robotics}
        \item Ranking de Google de revistas y conferencias de Visión por Computadora: \url{https://scholar.google.com/citations?view_op=top_venues&hl=en&vq=eng_computervisionpatternrecognition}
        \item Scimagojr: Conocer si una revista es Q1, Q2, Q3 o Q4 \url{https://www.scimagojr.com/}
    \end{itemize}

\section{Tratamiento de Figuras}
%
\begin{itemize}
    \item Las figuras deben estar vectorizadas con inkscape\footnote{\url{https://inkscape.org/}} o descargarlas ya vectorizadas (\lstinline{pdf} vectorizado o \lstinline{svg}).
    \item Las imágenes o gráficos que no sean fotografías siempre deben estar en formato \lstinline{svg} y ser exportadas \lstinline{pdf}. Es la imagen en formato \lstinline{pdf} la que luego se incluye en {\LaTeX}.
    
    \begin{figure}[!htbp]
        \centering
        \includegraphics[width=0.5\linewidth]{./images/bitmap_vs_svg.pdf}
        \caption{Diferencias entre imagen raster (jpg, png, tif, entre otros) y vecorizada (pdf vectorizado o svg).}
        \label{fig:bitmap_vs_svg}
    \end{figure}

    \item En inkscape es posible agregar código {\LaTeX} utilizando plugin \lstinline{textext}\footnote{\url{https://textext.github.io/textext/}}.
    \item En inkscape también se puede poner texto ``normal'' con la tipografía de {\LaTeX} por defecto que es CMU Serif  (Computer Modern Unicode Serif). Para esto se debe instalar primero el paquete \lstinline{sudo apt-get install fonts-cmu}
    \item En el texto, las figuras siempre se referencian de la misma manera. Ej: La Figura XXX ... Es decir, se las antepone únicamente con la palabra ``Figura''. Observar que la F es mayúscula, y XXX resulta de usar \textbackslash ref\{fig\:nombre\_figura\}.
    \item Todos los captions de las figuras y tablas deben terminar con punto (.)
    \item Para extraer gráficos vectorizados de un paper (\lstinline{pdf}), basta con abrir el archivo \lstinline{pdf} con inkscape.
    \item Cuando una imagen es extraída de un paper sin hacerle modificaciones se debe poner al final del caption la oración: Imagen extraída de [CITA DEL PAPER]. En cambio, si la figura fue modificada o adaptada se debe poner: Imagen adaptada de [CITA DEL PAPER].
    \item Buscar imágenes en el google con el keyword \lstinline{filetype: pdf} o \lstinline{filetype: svg}
    \item Si una imagen es un screenshot o una fotografía, debe estar en formato \lstinline{png} o \lstinline{jpg} de alta calidad (300 dpi, o un máximo de resolución de $1920\times 1080$ píxeles).
\end{itemize}


\section{Tratamiento de referencias}
Las referencias deben estar formateadas adecuadamente para mantener la formalidad y facilitar su reutilización en otros documentos. Para agregar una cita se debe descargar el bibtex de la página oficial donde esta publicado el trabajo (IEEE, Elsevier, Springer, etc). Agregar el bibtex al archivo \lstinline{bibliography.bib}, y luego hacer algunas correcciones:
\begin{itemize}
    \item El keyword único de la referencia debe tener el formato: \lstinline{apellido-primer-autor-año-primera-palabra-del-titulo}. Ej: \lstinline{davison2007monoslam}
    \item Completar el nombre de los autores, casi siempre estan con siglas los nombres, y las letras con acentos no aparecen. Ej: A. Davison -> Andrew Davison o Davison, Andrew
    \item Usar el Keyword definido de la conferencia o revista en el campo ``proceedings'' o ``journal''. Ej: proceedings = ICRA. Acá ICRA es una keyword que es reemplazada por el string ``IEEE Intl. Conf. on Robotics and Automation (ICRA)''
    \item Rodear al título con doble llave para que se mantengan todas las mayúsculas. Ej: title = \{ LALALA \} $\rightarrow$ \{\{ LALALA \}\}
    \item Le agrego un guión extra a las páginas, porque sino no se visualiza bien. Ej: 234-242 $\rightarrow$ 234--242
    \item Siempre debe estar el DOI y/o ISBN de la cita cuando esten disponibles.
    \item Un ejemplo de una referencia formateada adecuadamente es \cite{grisetti2010tutorial}
\end{itemize}


\section{Algoritmos}

Se debe utilizar el entorno \lstinline{algorithmic}.

Ejemplo:
    \begin{algorithmic}[1]
    \Procedure{ExtendedKalmanFilter}{$\mu_{t-1}, \covariance_{t-1}, \controlCommand_{t}, \observation_{t}$}
        \State $\overline{\mu}_{t} = \motionModelFunction{\controlCommand_{t}, \mu_{t-1}}$
        \State $\overline{\covariance}_{t} = \motionModelJacobian_{t} \covariance_{t-1} \motionModelJacobian_{t}^{\top}+\motionParametersCovariance_{t}$
        \Statex
        \State $\kalmanGain_{t} = \overline{\covariance}_{t} \observationModelJacobian_{t}^{\top} (\observationModelJacobian_{t} \overline{\covariance}_{t}  \observationModelJacobian_{t}^{\top} + \observationModelCovariance_{t})^{-1} $
        \State $\mu_{t} = \overline{\mu}_{t} + \kalmanGain_{t} (\observation_{t} - \observationModelFunction{\overline{\mu}_{t}})$
        \State $\covariance_{t} =  (I - \kalmanGain_{t} \observationModelJacobian_{t}) \overline{\covariance}_{t}$
        \State \Return $\mu_{t}, \covariance_{t}$
    \EndProcedure
    \end{algorithmic}

\section{Código}
Para poner código o partes de código se debe utilizar el paquete de {\LaTeX} \lstinline{listings}: \url{https://ctan.org/pkg/listings}.

\section{Escritura de Publicaciones}

\begin{itemize}
    \item Guía de escritura de papers del Prof. Cyrill Stachniss \url{https://www.youtube.com/watch?v=QYbAvOPcy0s}
\end{itemize}

\section{Herramientas Útiles}

\begin{itemize}
    \item aria2c: gestor de descargas por terminal \url{https://aria2.github.io/}.
    
    Ejemplo de uso: \lstinline{aria2c -x8 -s8 <link_de_descarga>}
    \item ROS2 (Robotics Operating System): framework de robótica
    \item Gazebo: simulador
    \item Git: versionador
    \item Docker: contenedor
    \item {\LaTeX}/Lyx: escritura de informes o papers
    \item VScode: IDE de desarrollo
    \item Yed: diagramas \url{https://www.yworks.com/products/yed}
    \item inkscape: gráficos vectorizados + plugin \url{https://textext.github.io/textext/}
    \item pdfpc: visualizador de slides (hechas en {\LaTeX}) \url{https://pdfpc.github.io/}
    \item Visualizar Rotaciones \url{https://articulatedrobotics.xyz/tools/rotation-calculator}
\end{itemize}


\printbibliography

\end{document}