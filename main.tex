\documentclass[a4paper,	11pt]{article}
\usepackage[spanish,es-tabla]{babel} % para acentos en español
\usepackage[utf8]{inputenc} % para poder copiar acentos desde el pdf
\usepackage[T1]{fontenc} % para poder copiar acentos desde el pdf
\usepackage{amsmath}
\usepackage{comment}
\usepackage[hidelinks]{hyperref}
\usepackage{color}
\usepackage{siunitx}
\setlength{\marginparwidth}{2cm}
\usepackage{graphicx}
\usepackage[export]{adjustbox}
\usepackage{multirow}
\usepackage{tablefootnote}

%\setlength{\marginparwidth}{2cm}
\usepackage{todonotes}

% changes margins
\usepackage[a4paper, total={7.5in, 10.6in}]{geometry}

\begin{document}

\renewcommand\floatpagefraction{.9}
\renewcommand\topfraction{.9}
\renewcommand\bottomfraction{.9}
\renewcommand\textfraction{.1}
\setcounter{totalnumber}{50}
\setcounter{topnumber}{50}
\setcounter{bottomnumber}{50}

\title{Tips Escritura Informes}
\author{Taihú Pire}


\tableofcontents

\maketitle


Este documento es una guía para hacer propuestas, informes, tesinas, tesis, publicaciones y presentaciones.

\section{Estructura general de un Abstract}
Cada oración del abstract tiene que decir algo puntual:\\
La 1ra oración puede ser la que pone contexto.\\
La 2da oración habla sobre el problema que se intenta abordar.\\
La 3ra oración describe el método.\\
La 4ta oración de cómo el método es validado.\\
(El abstract de la tesina puede ser un poco más extenso)

\section{Estructura General de una Tesina}
Portada\\
Abstract\\
1. Introducción\\
1.1 Motivación\\
1.2 Objetivos\\
1.3 Trabajos publicados durante la tesina (si los hubiere)\\
1.4 Estructura de la tesina\\
2. Trabajo Relacionado\\
3. Conceptos Previos\\
4. Método\\
5. Experimentos <- Este capítulo es el primero que debe escribirse\\
6. Conclusiones\\
6.1 Trabajo Futuro\\
Apéndices (si los hubiere)\\
Referencias

\subsection{Comentarios sobre Trabajo Relacionado}
\begin{itemize}
    \item Explicar los trabajos relacionados que se referencian: características principales, ventajas desventajas, etc.
\end{itemize}


\subsection{Comentarios generales}
%
\begin{itemize}
    \item Ser detallado y claro en el texto, repetir palabras hace que sea fácil de leer.

    \item Redactar de manera formal: no utilizar "etc" por ejemplo, en vez utilizar ``entre otros''.

    \item Mantener una terminología. Ejemplo: usar V-SLAM o SLAM visual, no ambas.

    \item Cada capítulo debe tener un texto introductorio que cuente de qué va el capítulo, qué se presenta o describe.
    
    \item Todas las ecuaciones (incluso las que están centradas) deben terminar con punto (.) o con coma (,). En el caso de coma y punto y seguido no debe haber sangría en la oración que sigue a la ecuación.

    \item Usar itálica para texto en inglés.

    \item Para un término en inglés Ej: \emph{keyframe} utilizar itálica únicamente en la primera ocurrencia de la palabra, luego no se usa más itálica para ese término.

    \item Para las unidades o números con unidades utilizar el paquete siunitx de latex. Ej: \SI{5}{\meter} o \si{\meter} para unidades solas.

    \item Remover citas que no se referencien en el texto, esto hace que el .bib quede más limpio.
\end{itemize}

\subsection{Tratamiento de Figuras}
%
\begin{itemize}
    \item Las figuras deben estar vectorizadas con inkscape o descargarlas ya vectorizadas (pdf vectorizado o svg)
    \item Para los gráficos vectorizados propongo usar inkscape tiene PPA para instalarlo. Agregar el plugin textext de inkscape para poner texto en latex en las figuras.
    \item Las figuras siempre se referencian de la misma manera. Ej: La Figura XXX ... Es decir, se las antepone únicamente con la palabra "Figura". Observar que la F es mayúscula, y XXX resulta de usar \ref{fig:nombre_figura}.
    \item Todos los captions de las figuras y tablas deben terminar con punto (.)
\end{itemize}


\section{Tratamiento de referencias}
Si bien el formato de las citas no es lo importante ahora, está bueno formatearlas bien para luego sea más fácil su uso y aparte son más fáciles de seguir y de referenciar. Para agregar una cita se debe descargar el bibtex de la página oficial donde esta publicado el trabajo (IEEE, Elsevier, Springer, etc). Agregar el bibtex al .bib y luego hacer algunas correcciones:
\begin{itemize}
    \item La referencia de la cita intento que tenga el formato: apellido-primer-autor-año-primera-palabra-del-titulo. Ej: davison2007monoslam
    \item Completar el nombre de los autores, casi siempre estan con siglas los nombres, y las letras con acentos no aparecen. Ej: A. Davison -> Andrew Davison o Davison, Andrew
    \item Usar el Keyword definido de la conferencia o revista en el campo ``proceedings'' o ``journal''. Ej: proceedings = ICRA. Acá ICRA es una keyword que es reemplazada por el string ``IEEE Intl. Conf. on Robotics and Automation (ICRA)''
    \item Rodear al título con doble llave para que se mantengan todas las mayúsculas. Ej: title = \{ LALALA \} $\rightarrow$ \{\{ LALALA \}\}
    \item Le agrego un guión extra a las páginas, porque sino no se visualiza bien. Ej: 234-242 $\rightarrow$ 234--242
    \item Siempre debe estar el DOI y/o ISBN de la cita cuando esten disponibles.
\end{itemize}
\end{document}