% para la bibliografía se requiere biber y configurar texstudio

% Latex packages
\usepackage[utf8]{inputenc}
\usepackage[T1]{fontenc} % para copiar acentos en español del pdf y permite acentos en las notas
\usepackage[spanish]{babel}
\usepackage[per-mode = symbol]{siunitx} % para manejar las unidades
\usepackage{multimedia} % to add videos with \movie command
\usepackage{multirow}
\usepackage{graphicx}
\usepackage{xcolor}
\usepackage{amsmath} % bmatrix
\usepackage[makeroom]{cancel} % \cancel to cancel terms in math equations
\renewcommand{\CancelColor}{\color{red}} % set red color for \cancel command
\usepackage[caption=false]{subfig} % caption = false elimina la palabra "Figura" del caption
\usepackage{import} % para el comando import (se usa para pdf_tex)
\captionsetup[subfigure]{labelformat=empty} % remover el indice del caption de la subfigura
\usepackage{booktabs} % \toprule \midrule \bottomrule
\usepackage[backend=biber]{biblatex} % set biber to format references. Must configure Biber in Texstudio
\usepackage{csquotes} % to remove warning triggered by biblatex and babel
\usepackage{algorithm} % to put captions to the algorithmics environmets
\usepackage{algpseudocode} % to write algorithm
\usepackage{tikz} % to use tikz
\usepackage[export]{adjustbox} %valign in subfloat
\usepackage{colortbl} % to paint cells in a table
\usepackage{todonotes} % add latex todo notes

% Color commands for annotations
\newcommand\TODO[1]{\textbf{\textcolor{red}{#1}}} %  TODO notes

% Graphic paths
\graphicspath{{./images/}}

% listings configuration for C code
\usepackage{listings} % code
\definecolor{commentgreen}{RGB}{2,112,10}
\definecolor{eminence}{RGB}{108,48,130}
\definecolor{weborange}{RGB}{255,165,0}
\definecolor{frenchplum}{RGB}{129,20,83}

\lstset{ % spanish characters for listings package
	inputencoding=latin1,
    columns=fullflexible,
	breaklines=true,
	tabsize=2,
	showstringspaces=false,
	basicstyle=\ttfamily,
	backgroundcolor=\color{lightgray}, % Choose background color
	literate={á}{{\'a}}1
	{ã}{{\~a}}1
	{é}{{\'e}}1
	{ó}{{\'o}}1
	{í}{{\'i}}1
	{ñ}{{\~n}}1
	{¡}{{!`}}1
	{¿}{{?`}}1
	{ú}{{\'u}}1
	{Í}{{\'I}}1
	{Ó}{{\'O}}1
    {-}{-}1
}

\lstdefinestyle{cpp}{ % spanish characters for listings package
    language=C++,
   	commentstyle=\color{commentgreen},
    keywordstyle=\color{eminence},
    stringstyle=\color{red},
    emph={int,char,double,float,unsigned,void,bool},
    emphstyle={\color{blue}}
}

\lstdefinestyle{bash}{ % spanish characters for listings package
	language=Bash
}

\lstdefinestyle{xml}{
	language=XML,
	morekeywords={encoding,xs:schema,xs:element,xs:complexType,xs:sequence,xs:attribute}
}

\lstdefinestyle{cmake}{
	language=make, % there is no cmake support in listings
}

\lstdefinestyle{python}{
    language=python,
}


%%%%% PARA QUE EN LAS TABLAS SE PUEDA PONER UN SALTO DE LINEA DENTRO DE UNA CELDA
\newcommand{\specialcell}[2][c]{%
    \begin{tiny}
        \begin{tabular}[#1]{@{}c@{}}#2\end{tabular}  
    \end{tiny}
}
%%%%%%%%%%%%%%%%%%%%%%%%%%%%%%%%%%%%%%%%%%%%%%%%%%%%%%%%%%%%%%%%%%%%%%%%

%%%%% PARA QUE LAS TABLAS TENGAN TODAS LAS COLUMNAS CENTRADAS Y DE IGUAL TAMAÑO
\usepackage{tabularx}
\renewcommand{\tabularxcolumn}[1]{>{\centering\arraybackslash}m{#1}}
%%%%%%%%%%%%%%%%%%%%%%%%%%%%%%%%%%%%%%%%%%%%%%%%%%%%%%%%%%%%%%%%%%%%%%%%

